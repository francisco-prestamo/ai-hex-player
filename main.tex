\documentclass{article}
\usepackage[spanish]{babel}
\usepackage{amsmath}
\usepackage{enumitem}
\usepackage{hyperref}
\usepackage{xcolor}
\usepackage{listings}

\begin{document}

\title{Implementación de Minmax para Hex}
\author{Francisco Prestamo Bernardez}
\date{}
\maketitle

\section{Descripción General}
Un agente de IA simple para jugar al juego de tablero Hex en el cual se combinan diferentes estrategias y algoritmos para realizar movimientos óptimos dentro de restricciones de tiempo.

\section{Componentes Principales}

\subsection{Integración Python-C}
\begin{itemize}
    \item Utiliza \texttt{ctypes} para interactuar entre Python y C
    \item Compila código C como biblioteca compartida para rendimiento
    \item Gestiona memoria y transferencia de datos entre lenguajes
\end{itemize}

\subsection{Sistema de Toma de Decisiones}
\begin{itemize}
    \item Implementa algoritmo Minimax con poda alfa-beta
    \item Profundidad de búsqueda configurable (predeterminado = 4)
    \item Búsqueda limitada por tiempo mediante un hilo de tiempo de espera separado
    \item Los primeros movimientos se manejan con estrategias especiales
\end{itemize}

\subsection{Evaluación del Tablero}

La evaluación del tablero se basa en una combinación ponderada de cuatro heurísticas principales:

\begin{enumerate}
    \item Rango de Conectividad
    \begin{itemize}
        \item Corresponde al mayor rango alcanzado en el eje relevante (horizontal para el jugador 1, vertical para el jugador 2) dentro del conjunto disjunto del que forma parte la celda.
        \item Representa la expansión máxima del grupo conectado del jugador.
    \end{itemize}

    \item Proximidad a Conexión Óptima
    \begin{itemize}
        \item Utiliza una métrica basada en el algoritmo de Dijkstra para estimar la distancia mínima desde la celda a ambos lados del tablero.
        \item Una menor suma de distancias se considera más favorable.
    \end{itemize}

    \item Valor de Bloqueo
    \begin{itemize}
        \item Mide la cantidad de posiciones enemigas potencialmente bloqueadas desde la celda evaluada.
        \item Favorece los movimientos que interrumpen la conectividad del oponente.
    \end{itemize}

    \item Conexiones Estratégicas
    \begin{itemize}
        \item Evalúa la capacidad de la celda para conectarse con otras piezas propias de valor estratégico.
        \item Favorece la formación de caminos robustos o conexiones críticas.
    \end{itemize}
\end{enumerate}


\subsection{Optimizaciones de Rendimiento}
\begin{itemize}
    \item Ordenamiento de movimientos usando QuickSort
    \item Uso de Disjoint Set para detección de victoria
    \item Terminación anticipada con poda alfa-beta
    \item Exploración de movimientos basada en prioridades
\end{itemize}


\section{Uso}
La IA puede utilizarse creando una instancia de la clase \texttt{AstroBot} con un ID de jugador (1 o 2) y llamando a su método \texttt{play()} con:
\begin{itemize}
    \item Estado actual del tablero
    \item Límite de tiempo para el cálculo del movimiento
\end{itemize}
Nota: Es necesario estar en la carpeta del player y el archivo c al mismo tiempo para la ejecución del código

\end{document}